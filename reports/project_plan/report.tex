\documentclass[10pt,letterpaper]{article}
\usepackage{fullpage}
\usepackage{amsmath}
\usepackage{enumitem}
\usepackage{amssymb}
\usepackage{natbib}
\usepackage{graphicx}

\setlist[enumerate]{itemsep=0mm}

\usepackage[top=0.8in,bottom=0.8in,left=0.8in,right=0.8in]{geometry}

\newcommand{\somespace}{\vspace{0.075in}}

\title{\textbf{PROJECT PLAN}}

\author{\vspace{-0.6in}}

\date{\vspace{-0.6in}}

\begin{document}

\maketitle

\noindent \textbf{1. Students' name and Purdue e-mail.}
\somespace

\begin{enumerate}[label=\alph*.]
    \item Nikita Rajaneesh | nrajanee@purdue.edu
    \item Swaraj Bhaduri | sbhadur@purdue.edu
    \item Utkarsh Jain | jain192@purdue.edu
    \item Ishaan Saxena | isaxena@purdue.edu
\end{enumerate}

\somespace
\noindent \textbf{2. Definition of the problem, possibly relevant to your interests.}
\somespace

Analysis of Mushroom Species Data by classification into groups of poisonous and edible
based on features such as caps, odor, stalks, etc. and comparing different classification
models and algorithms to determine which is best suited to this problem.

\somespace
\noindent \textbf{3. Description of the dataset (or datasets) to be used.}
\somespace

Size of dataset (before encoding): $ (n=8124, d=22) $\\

Size of dataset (after encoding): $ (n=8124, d=107) $\\

Classes: edible=e, poisonous=p (y-values)\\

Attribute Information and Encoding:
\begin{enumerate}
    \item \underline{Nominal Categorical Variables:}\\
    These variables will be encoded as binary one-hot features. As a result, each feature
    in this category would be replace by the $ k $ features in the encoded dataset if the
    feature has $ k $ possible values. These featues include:
    \begin{enumerate}[label=\roman*.]
        \item \textbf{cap-shape}:
            bell=b,
            conical=c,
            convex=x,
            flat=f,
            knobbed=k,
            sunken=s
        \item \textbf{cap-surface}:
            fibrous=f,
            grooves=g,
            scaly=y,
            smooth=s
        \item \textbf{cap-color}:
            brown=n,
            buff=b,
            cinnamon=c,
            gray=g,
            green=r,
            pink=p,
            purple=u,
            red=e,
            white=w,
            yellow=y
        \item \textbf{bruises}:
            bruises=t,
            no=f
        \item \textbf{odor}:
            almond=a,
            anise=l,
            creosote=c,
            fishy=y,
            foul=f,
            musty=m,
            none=n,
            pungent=p,
            spicy=s
        \item \textbf{gill-attachment}:
            attached=a,
            descending=d,
            free=f,
            notched=n
        \item \textbf{gill-color}:
            black=k,
            brown=n,
            buff=b,
            chocolate=h,
            gray=g,
            green=r,
            orange=o,
            pink=p,
            purple=u,
            red=e,
            white=w,
            yellow=y
        \item \textbf{stalk-root}:
            bulbous=b,
            club=c,
            cup=u,
            equal=e,
            rhizomorphs=z,
            rooted=r,
            missing=?
        \item \textbf{stalk-surface-above-ring}:
            fibrous=f,
            scaly=y,
            silky=k,
            smooth=s
        \item \textbf{stalk-surface-below-ring}:
            fibrous=f,
            scaly=y,
            silky=k,
            smooth=s
        \item \textbf{stalk-color-above-ring}:
            brown=n,
            buff=b,
            cinnamon=c,
            gray=g,
            orange=o,
            pink=p,
            red=e,
            white=w,
            yellow=y
        \item \textbf{stalk-color-below-ring}:
            brown=n,
            buff=b,
            cinnamon=c,
            gray=g,
            orange=o,
            pink=p,
            red=e,
            white=w,
            yellow=y
        \item \textbf{veil-type}:
            partial=p,
            universal=u
        \item \textbf{veil-color}:
            brown=n,
            orange=o,
            white=w,
            yellow=y
        \item \textbf{ring-type}:
            cobwebby=c,
            evanescent=e,
            flaring=f,
            large=l,
            none=n,
            pendant=p,
            sheathing=s,
            zone=z
        \item \textbf{spore-print-color}:
            black=k,
            brown=n,
            buff=b,
            chocolate=h,
            green=r,
            orange=o,
            purple=u,
            white=w,
            yellow=y
        \item \textbf{habitat}:
            grasses=g,
            leaves=l,
            meadows=m,
            paths=p,
            urban=u,
            waste=w,
            woods=d
    \end{enumerate}
    \item \underline{Ordinal Categorical Variables:}\\
    These variables will be encoded in place by encoding labels, as the data here has
    ordinal meaning to it. These variables include:
    \begin{enumerate}[label=\roman*.]
        \item \textbf{gill-spacing}:
            close=c$\to$0,
            crowded=w$\to$1,
            distant=d$\to$2
        \item \textbf{gill-size}:
            broad=b$\to$0,
            narrow=n$\to$1
        \item \textbf{stalk-shape}:
            enlarging=e$\to$0,
            tapering=t$\to$1
        \item \textbf{ring-number}:
            none=n$\to$0,
            one=o$\to$1,
            two=t$\to$2
        \item \textbf{population}:
            abundant=a$\to$0,
            clustered=c$\to$1,
            numerous=n$\to$2,
            scattered=s$\to$3,
            several=v$\to$4,
            solitary=y$\to$5
    \end{enumerate}
\end{enumerate}

\somespace
\noindent \textbf{4. URL where the above dataset(s) is(are) available.}
\somespace

https://www.kaggle.com/uciml/mushroom-classification

\somespace
\noindent \textbf{5. Which machine learning algorithm(s) is(are) going to be used?}
\somespace

We are looking to compare the performance of different algorithms, including
Support Vector Machines, Adaboost, and Logistic Regression using different metrics. We
will also use feature selection to ensure additional complexity control and gain
interpretability.

\somespace
\noindent \textbf{6. Cross-validation technique (e.g., training/validation/testing, k-fold cross-validation, bootstrapping)}
\somespace

k-Fold Cross Validation would be used with about 5 to 10 folds.

\somespace
\noindent \textbf{7. Which hyperparameter(s) is(are) going to be tuned.}
\somespace

Hyperparameters of the chosen algorithm would be tuned after it is determined. For instance,
if a Support Vector Machine is chosen, the value of the penalty parameter $ C $ for the
error terms (slack variables) will be adjusted.

\somespace
\noindent \textbf{8. Description of the experimental results, e.g., plots of number of samples versus accuracy (you can use different subsets of the same dataset), regularization parameter versus accuracy, ROC curves, plots of different datasets, etc.}
\somespace

Graphs for Hyperparameter tuning would include hyperparameter value vs accuracy, plots of
descion boundary and margins obtained by different hyperparameter values.

We will add plots for correlations between features, and correlations between features and
class labels, etc. for feature selection.

To compare models (and different kernels), we will use the sensitivity vs. specificity
curves, area under ROC curve, number of folds used vs. training accuracy for each model,
number of features (size of feature subset) vs accuracy.

\somespace
\noindent \textbf{9. Which programming language are you going to use?}
\somespace

Python-2.7

\somespace

\end{document}
