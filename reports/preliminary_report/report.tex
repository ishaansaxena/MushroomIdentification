\documentclass[fleqn]{article}
\usepackage{fullpage}
\usepackage{amsmath}
\usepackage{enumitem}
\usepackage{amssymb}
\usepackage{listings}
\usepackage{hyperref}
\usepackage{tikz}
\usepackage{algpseudocode}
\usepackage{algorithm}
\usepackage{multicol}

\renewcommand{\vec}[1]{\mathbf{#1}}

\DeclareMathOperator*{\argmax}{arg\,max}
\usetikzlibrary{arrows}
\lstset{basicstyle=\ttfamily, mathescape}
\graphicspath{{./img/}}

\title{Preliminary Report - Edibility of Mushroom Species}
\author{
    \begin{tabular}{cccc}
            Ishaan Saxena      & Nikita Rajaneesh    & Swaraj Bhaduri     & Utkarsh Jain       \\
            isaxena@purdue.edu & nrajanee@purdue.edu & sbhadur@purdue.edu & jain192@purdue.edu
    \end{tabular}
}

\begin{document}
    \maketitle

    % 1
    \section{Introduction to the Problem}

    % 1.1
    \subsection{Definition of the Problem}

    Given a dataset $ \mathcal{D} $ with $ n=8124 $ samples where each sample represents a
    mushroom with features being the observations about the characterestics of the mushrooms
    such as odor, color, etc., we aim to test and compare various supervised learning models
    for the problem of classifying each sample into either poisonous or edible. Further,
    we will optimize the Hyperparameters of the model which initially performs the best
    on the dataset.\footnote{This dataset can be found at
    \href{https://www.kaggle.com/uciml/mushroom-classification}
    {https://www.kaggle.com/uciml/mushroom-classification}.}

    % 1.2
    \subsection{Data Description}

    We are given $ \mathcal{D} $ with $ n=8124 $ samples wherein each sample has the
    following 22 features (excluding the class label).

    \begin{multicols}{3}
        \begin{enumerate}
            \item cap-shape
            \item cap-surface
            \item cap-color
            \item bruises
            \item odor
            \item gill-attachment
            \item gill-color
            \item stalk-root
            \item stalk-surface-above-ring
            \item stalk-surface-below-ring
            \item stalk-color-above-ring
            \item stalk-color-below-ring
            \item veil-type
            \item veil-color
            \item ring-type
            \item spore-print-color
            \item habitat
            \item gill-spacing
            \item gill-size
            \item stalk-shape
            \item ring-number
            \item population
        \end{enumerate}
    \end{multicols}

    These features have been further enumerated in Appendix A.

    % 1.3
    \subsection{Encoding the Data}

    Note that all the features in our dataset are categorical variables. As a result, to
    proceed with evaluation of model performance, we must first encode these variables
    into numerical/binary values.

    % Appendices
    % Appendix
    \newpage
    \section*{\underline{Appendix A}}

\end{document}
