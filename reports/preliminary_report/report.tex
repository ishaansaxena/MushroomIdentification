\documentclass[fleqn]{article}
\usepackage{fullpage}
\usepackage{amsmath}
\usepackage{enumitem}
\usepackage{amssymb}
\usepackage{listings}
\usepackage{hyperref}
\usepackage{tikz}
\usepackage{algpseudocode}
\usepackage{algorithm}
\usepackage{multicol}

\renewcommand{\vec}[1]{\mathbf{#1}}

\DeclareMathOperator*{\argmax}{arg\,max}
\usetikzlibrary{arrows}
\lstset{basicstyle=\ttfamily, mathescape}
\graphicspath{{./img/}}

\usepackage[utf8]{inputenc}

% Default fixed font does not support bold face
\DeclareFixedFont{\ttb}{T1}{txtt}{bx}{n}{12} % for bold
\DeclareFixedFont{\ttm}{T1}{txtt}{m}{n}{12}  % for normal

% Custom colors
\usepackage{color}
\definecolor{deepblue}{rgb}{0,0,0.5}
\definecolor{deepred}{rgb}{0.6,0,0}
\definecolor{deepgreen}{rgb}{0,0.5,0}

\usepackage{listings}

% Python style for highlighting
\newcommand\pythonstyle{\lstset{
language=Python,
basicstyle=\tt,
otherkeywords={self},             % Add keywords here
keywordstyle=\tt\color{deepblue},
emph={MyClass,__init__},          % Custom highlighting
emphstyle=\tt\color{deepred},    % Custom highlighting style
stringstyle=\color{deepgreen},
frame=tb,                         % Any extra options here
showstringspaces=false            %
}}


% Python environment
\lstnewenvironment{python}[1][]
{
\pythonstyle
\lstset{#1}
}
{}

% Python for external files
\newcommand\pythonexternal[2][]{{
\pythonstyle
\lstinputlisting[#1]{#2}}}

% Python for inline
\newcommand\pythoninline[1]{{\pythonstyle\lstinline!#1!}}

\title{Preliminary Report - Edibility of Mushroom Species}
\author{
    \begin{tabular}{cccc}
            Ishaan Saxena      & Nikita Rajaneesh    & Swaraj Bhaduri     & Utkarsh Jain       \\
            isaxena@purdue.edu & nrajanee@purdue.edu & sbhadur@purdue.edu & jain192@purdue.edu
    \end{tabular}
}

% Document
\begin{document}
    \maketitle

    % 1
    \section{Introduction to the Problem}

    % 1.1
    \subsection{Definition of the Problem}

    Given a dataset $ \mathcal{D} $ with $ n=8124 $ samples where each sample represents a
    mushroom with features being the observations about the characterestics of the mushrooms
    such as odor, color, etc., we aim to test and compare various supervised learning models
    for the problem of classifying each sample into either poisonous or edible. Further,
    we will optimize the Hyperparameters of the model which initially performs the best
    on the dataset.\footnote{This dataset can be found at
    \href{https://www.kaggle.com/uciml/mushroom-classification}
    {https://www.kaggle.com/uciml/mushroom-classification}.}

    % 1.2
    \subsection{Data Description}

    We are given $ \mathcal{D} $ with $ n=8124 $ samples wherein each sample has the
    following 22 features (excluding the class label).

    \begin{multicols}{3}
        \begin{enumerate}
            \item cap-shape
            \item cap-surface
            \item cap-color
            \item bruises
            \item odor
            \item gill-attachment
            \item gill-color
            \item stalk-root
            \item stalk-surface-above-ring
            \item stalk-surface-below-ring
            \item stalk-color-above-ring
            \item stalk-color-below-ring
            \item veil-type
            \item veil-color
            \item ring-type
            \item spore-print-color
            \item habitat
            \item gill-spacing
            \item gill-size
            \item stalk-shape
            \item ring-number
            \item population
        \end{enumerate}
    \end{multicols}
    \noindent
    These features have been further enumerated in Appendix A.

    % 1.3
    \subsection{Encoding the Data}

    Note that all the features in our dataset are categorical variables. As a result, to
    proceed with evaluation of model performance, we must first encode these variables
    into numerical/binary values.\\

    We need to deal with two kinds of categorical variables when encoding the features
    into numerical data. These are ordinal categorical variables and nominal categorical
    variables.\footnote{Information on which features are which kind of categorical
    variables can be found in Appendix A.} We will use different techniques to encode
    both of these kinds of categorical variables as they inherently represent different
    kinds of categorical data.\\

    To encode nominal categorical variables, we will use one-hot binary features. For
    instance, if a feature $ f $ from a feature set $ \mathcal{F} $ has $ k $ different
    categorical values, we can create $ k $ different binary features for each feature
    $ f $ of this kind. This is done because the values of each such feature do not hold
    any ordinal information, in that there should not be different weights for having a
    specific value of a specific feature.\\

    The ordinal categorical variables, on the other hand, have been encoded as numerical
    labels, as these informations contain valueable information about the 'scale' of a
    certain feature. For instance, if a feature $ f $ from a feature set $ \mathcal{F} $
    has $ k $ different features, it would be changed into numeric values
    $ i \in 0,1,...,k-1$.\\

    \noindent
    The following code segment is used to encode the data.\footnote{This can be found in
    the data.py file.}\\

    \begin{python}
def encode(df):
    # Encode Ordinal Variables
    ordinal_columns = ['gill-spacing', 'gill-size',
            'stalk-shape', 'ring-number', 'population', 'class']
    columns = ordinal_columns[:]

    for column in columns:
        df[column] = df[column].astype('category')

        columns = df.select_dtypes(['category']).columns
        df[columns] = df[columns].apply(lambda x: x.cat.codes)

    # Encoding Nominal Variables
    columns = ordinal_columns[:]

    for column in df:
        if column not in columns:
            dummies = pd.get_dummies(df.pop(column))
            column_names = [column + "_" + x for x in dummies.columns]
            dummies.columns = column_names
            df = df.join(dummies)

    return df
    \end{python}
















































    % Appendices
    % Appendix
    \newpage
    \section*{\underline{Appendix A: Data Description}}
    Classes: edible=e, poisonous=p (y-values)\\

    Size of dataset (before encoding): $ (n=8124, d=22) $

    Size of dataset (after encoding): $ (n=8124, d=107) $

    Attribute Information and Encoding:
    \begin{enumerate}
        \item \underline{Nominal Categorical Variables:}\\
        These variables will be encoded as binary one-hot features. As a result, each feature
        in this category would be replace by the $ k $ features in the encoded dataset if the
        feature has $ k $ possible values. These featues include:
        \begin{enumerate}[label=\roman*.]
            \item \textbf{cap-shape}:
                bell=b,
                conical=c,
                convex=x,
                flat=f,
                knobbed=k,
                sunken=s
            \item \textbf{cap-surface}:
                fibrous=f,
                grooves=g,
                scaly=y,
                smooth=s
            \item \textbf{cap-color}:
                brown=n,
                buff=b,
                cinnamon=c,
                gray=g,
                green=r,
                pink=p,
                purple=u,
                red=e,
                white=w,
                yellow=y
            \item \textbf{bruises}:
                bruises=t,
                no=f
            \item \textbf{odor}:
                almond=a,
                anise=l,
                creosote=c,
                fishy=y,
                foul=f,
                musty=m,
                none=n,
                pungent=p,
                spicy=s
            \item \textbf{gill-attachment}:
                attached=a,
                descending=d,
                free=f,
                notched=n
            \item \textbf{gill-color}:
                black=k,
                brown=n,
                buff=b,
                chocolate=h,
                gray=g,
                green=r,
                orange=o,
                pink=p,
                purple=u,
                red=e,
                white=w,
                yellow=y
            \item \textbf{stalk-root}:
                bulbous=b,
                club=c,
                cup=u,
                equal=e,
                rhizomorphs=z,
                rooted=r,
                missing=?
            \item \textbf{stalk-surface-above-ring}:
                fibrous=f,
                scaly=y,
                silky=k,
                smooth=s
            \item \textbf{stalk-surface-below-ring}:
                fibrous=f,
                scaly=y,
                silky=k,
                smooth=s
            \item \textbf{stalk-color-above-ring}:
                brown=n,
                buff=b,
                cinnamon=c,
                gray=g,
                orange=o,
                pink=p,
                red=e,
                white=w,
                yellow=y
            \item \textbf{stalk-color-below-ring}:
                brown=n,
                buff=b,
                cinnamon=c,
                gray=g,
                orange=o,
                pink=p,
                red=e,
                white=w,
                yellow=y
            \item \textbf{veil-type}:
                partial=p,
                universal=u
            \item \textbf{veil-color}:
                brown=n,
                orange=o,
                white=w,
                yellow=y
            \item \textbf{ring-type}:
                cobwebby=c,
                evanescent=e,
                flaring=f,
                large=l,
                none=n,
                pendant=p,
                sheathing=s,
                zone=z
            \item \textbf{spore-print-color}:
                black=k,
                brown=n,
                buff=b,
                chocolate=h,
                green=r,
                orange=o,
                purple=u,
                white=w,
                yellow=y
            \item \textbf{habitat}:
                grasses=g,
                leaves=l,
                meadows=m,
                paths=p,
                urban=u,
                waste=w,
                woods=d
        \end{enumerate}
        \item \underline{Ordinal Categorical Variables:}\\
        These variables will be encoded in place by encoding labels, as the data here has
        ordinal meaning to it. These variables include:
        \begin{enumerate}[label=\roman*.]
            \item \textbf{gill-spacing}:
                close=c$\to$0,
                crowded=w$\to$1,
                distant=d$\to$2
            \item \textbf{gill-size}:
                broad=b$\to$0,
                narrow=n$\to$1
            \item \textbf{stalk-shape}:
                enlarging=e$\to$0,
                tapering=t$\to$1
            \item \textbf{ring-number}:
                none=n$\to$0,
                one=o$\to$1,
                two=t$\to$2
            \item \textbf{population}:
                abundant=a$\to$0,
                clustered=c$\to$1,
                numerous=n$\to$2,
                scattered=s$\to$3,
                several=v$\to$4,
                solitary=y$\to$5
        \end{enumerate}
    \end{enumerate}

\end{document}
